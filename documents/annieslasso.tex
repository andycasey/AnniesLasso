%% This file is part of the Annie's Lasso project.
%% Copyright 2015 the authors.  All rights reserved.

\documentclass[12pt]{aastex}

\begin{document}

\title{The Cannon 2: A data-driven model for detailed chemical abunance analysis}
\author{AC, DWH, MKN, others}

\begin{abstract}
% context
In previous work, we have shown that it is possible to train a generative
probabilistic model for stellar spectra using a training set of stars, each with known
parameters ($\logg$ and $\Teff$) and chemical abundance, and then use that model to infer labels for
unlabeled stars, even stars with lower signal-to-noise observations.
% aims
Here we ask whether this is possible when the dimensionality of the chemical
abundance space is large (15 abundances: Mg, Fe, etc, etc, Al)
and the model is non-linear in its response to abundance and parameter changes.
% method
We adopt ideas from compressed sensing to limit overall model complexity (number
of non-zero parameters) while retaining model freedom.
The training set is a set of YYY red-giant stars with high signal-to-noise
spectroscopic observations and stellar parameters and abundances taken from the
\apogee\ Survey.
% results
We find that we can successfully train and use a model with 17 stellar labels.
Cross-validation shows that the model does a good job of inferring all 17 labels
(with the exception of XXX), even when we degrade the signal-to-noise of the
validation set by discarding some of the spectroscopic observing time.
The model dependencies make sense; the derivatives of the spectral mean model
with respect to abundances correlate well with known atomic lines.
We deliver open-source code and also stellar parameters and 15 abundances for a
set of ZZZ stars.
\end{abstract}

Hello World!

\end{document}
